\documentclass[a4paper, 12pt]{article} 
\setlength{\textheight}{24.0cm}
\setlength{\textwidth}{16cm}
\setlength{\parindent}{0.0cm}
\setlength{\topmargin}{-1.0cm}
\setlength{\oddsidemargin}{0.0cm}
\renewcommand{\baselinestretch}{1.5}
    
\usepackage{graphicx}
\usepackage{amsmath}
\usepackage{diffcoeff}
\usepackage{amssymb}
\usepackage{subcaption} 

% Dash int definitions
\def\Xint#1{\mathchoice
{\XXint\displaystyle\textstyle{#1}}%
{\XXint\textstyle\scriptstyle{#1}}%
{\XXint\scriptstyle\scriptscriptstyle{#1}}%
{\XXint\scriptscriptstyle\scriptscriptstyle{#1}}%
\!\int}
\def\XXint#1#2#3{{\setbox0=\hbox{$#1{#2#3}{\int}$ }
\vcenter{\hbox{$#2#3$ }}\kern-.6\wd0}}
\def\ddashint{\Xint=}
\def\dashint{\Xint-}

\numberwithin{equation}{section}
    
\begin{document}
\thispagestyle{empty}
\Large
\vspace*{2cm}
\begin{center}
    \textbf{UNIVERSITY OF STRATHCLYDE}
\end{center}

\vspace*{1cm}
\begin{center}
    \textbf{DEPARTMENT OF\\ MATHEMATICS AND STATISTICS}
\end{center}

\vspace*{1cm}
\begin{center}
    \textbf{THE N-DIMENSIONAL WAVE EQUATION AND THE METHOD OF DESCENT}
\end{center}

\vspace*{1cm}
\begin{center}
    \textbf{by}
\end{center}

\vspace*{1cm}
\begin{center}
    \textbf{IOANA-TEODORA VADUVA}\\
    \textbf{201444379}
\end{center}

\vspace*{3cm}
\begin{center}
    \textbf{BSc Hons Mathematics}
    \textbf{2017/18}
\end{center}

\newpage
\Large

\vspace*{1cm}
\begin{center}
    \textbf{Statement of work in project}
\end{center}

\vspace*{1cm}
\begin{center}
    \parbox{14cm}{
        The work contained in this project is that of the author and where
        material from other sources has been incorporated full acknowledgement
        is made.}
    \end{center}
    
    \begin{center}
        \begin{tabbing}
            xxxxx\=xxxxxxxxxxxxxxx\= \kill
            \\
            \\
            \\
            \>Signed\>.........................................................\\
            \\
            \>Print Name\>.........................................................\\
            \\
            \>Date\>.........................................................\\
        \end{tabbing}
    \end{center}
    
    \vspace*{5cm}
    Supervised by Dr Marcus Waurick
    \newpage
    \normalsize
    \tableofcontents
    \newpage
    
\section{Introduction}
This project presents the wave equation and its solution in one, two, three and higher dimensions. The wave equation is a partial differential equation (PDE), an identity
that relates independent variables ($x, y...$), the dependent variable $u$ (a function of the independent variables) and the partial derivatives of $u$. The wave 
equation is a PDE of the second order, given by the highest appearing derivative, see \cite{Fol}.
\\

A guitar and a violin make different sounds, despite both being played by plucking the string. The wave equation allows an insight into how this happens through an
accurate model of the vibrating string. Musical instruments are not, however, the  only application of the wave equation, which can model accurately electromagnetic 
waves, waves in fluids, waves caused by volcanoes and earthquakes and even has a role in quantum mechanics. Historically, it is of importance and many representative 18th century
mathematicians: D'Alambert, Euler, Lagrange and Bernoulli, to name a few, had contributions to the development, modelling and solving of the wave equation, see \cite{Coc}.
\\

In this project, we focus our attention on the general wave equation which works in any number of dimensions. We begin by deriving it from physical principles, and then 
focus most of our attention on solving the problem in one, three and two dimensions, generalising our result to higher dimensions. We then finish by proving the uniqueness 
of our solution.

\section{Derivation of the Wave Equation}
In this section, we will derive the wave equation in one- and two-dimensions, showing a pattern that 
we can extend to higher dimensions. Note that the images have been created using \emph{Photoshop}, but their original version comes from \cite{Kr}.

\subsection{One-Dimension}
The wave equation arises from the movement of a string in a musical instrument, such as a guitar or a violin, see
\cite{BoyDiP}. This string is perfectly elastic, and
it is stretched between two supports, at $x=0$ and $x=L$. The elastic string is of length $L$, as
illustrated in Figure \ref{fig:1a}. At time $t=0$, the string is set into motion and left undisturbed to vibrate in the
vertical plane. For this derivation of the wave equation, we will simplify things by neglecting damping 
effects such as air resistance. We will obtain this using the Newtonian
mechanics of forces on a small part of the string, of length $\Delta x$, which lies between $x$ and 
$x+\Delta x$. The motion of this piece of string is very small and negligible in the horizontal plane,
and so we only have movement
in the vertical direction, see \cite{Kr}. Let $u(x,t)$ be the vertical displacement of point $x$ at the time $t$, 
$T(x,t)$ - the tension in the string that acts only in the direction tangent to the string, and $\rho$ the 
mass per unit length of the string. 

\begin{figure}[h]
    \begin{subfigure}[t]{0.5\textwidth} 
        \includegraphics[width=0.9\linewidth]{images/grafic-1.png} 
        \caption{String of length L}
        \label{fig:1a}
    \end{subfigure}
    \begin{subfigure}[t]{0.5\textwidth}
        \includegraphics[width=0.9\linewidth]{images/grafic-2.png}
        \caption{Components of tension}
        \label{fig:1b}
    \end{subfigure}     
\caption{}
\label{fig:1}
\end{figure}

In \cite{Tay}, \emph{Newton's Second Law} says that ``for any particle of mass $m$, the net force 
$\boldsymbol{F}$ on the particle is always equal to the mass $m$ times the particle's 
acceleration: $\boldsymbol{F} = m \boldsymbol{a}$.''  We apply this to our 
portion of the string $\Delta x$, so that the external
force, the tension at the ends of the string portion, equals to the product of the mass of the element and 
its acceleration at the mass centre. 
\\

Looking at the movement of the section of string, we see that there is no acceleration in the horizontal direction,  
as this movement is insignificant to the system. Therefore, we obtain the 
following equation governing the horizontal motion, by adding together the horizontal components of tension from Figure \ref{fig:1b}:

\begin {equation} \label{eq1}
    T(x+\Delta x,t)\cos{(\theta + \Delta \theta)}-T(x,t)\cos{\theta}=0
\end {equation}

The left-hand side of (\ref{eq1}) is independent of $x$ and we will denote it by $H(t)$.
\\

Looking at the vertical movement, we obtain a similar equation to (\ref{eq1}), with the distinction
that now there exists a vertical acceleration. This is given by $u_{tt} (\bar{x},t)$, where $u_{tt}$ is the second derivative
of the displacement with respect to time. Here,
 $\bar{x}$ is the coordinate of the centre of mass of the portion of string and $x<\bar{x}<x+\Delta x$. The mass of the 
 section $\Delta x$ is $\rho\Delta x$, and so the vertical movement is given by:

 \begin{equation} \label{eq2}
    T(x+\Delta x,t)\sin{(\theta + \Delta \theta)}-T(x,t)\sin{\theta}=\rho\Delta x u_{tt} (\bar{x},t).
 \end{equation}

 Usually, the weight of the string would also act vertically downwards, but we neglect it here. Similarly to the 
 horizontal case, we can denote the left-hand side of equation (\ref{eq2}) by $V(x,t)=T(x,t)$ and rewrite (\ref{eq2})
 as:

 \begin{equation} \label{eq3}
    \frac{V(x+\Delta x,t)-V(x,t)}{\Delta x}=\rho\Delta x u_{tt} (\bar{x},t)
 \end{equation}

 Taking the limit of (\ref{eq3}) as $\Delta x \rightarrow 0$ (the equivalent of making the string smaller and smaller)
 we get, from the limit definition of the derivative described in \cite{Spi}, that 

 \begin{equation*}
    \lim_{\Delta x \rightarrow 0}\frac{V(x+\Delta x,t)-V(x,t)}{\Delta x}=V_x(x,t),
 \end{equation*}

i.e. that 
\begin {equation} \label{eq4}
    V_x(x,t)=\rho\Delta x u_{tt} (x,t).
\end{equation}

We want to express equation (\ref{eq4}) in terms of $u$ entirely. Using the definition of the tangent function we see that the vertical movement is the
horizontal movement multiplied by the tangent $V(x, t)=H(t)\tan{\theta}$. As the derivative with respect to $x$ is the slope of
the tangent, as in \cite{Spi}, we have:

\begin{equation} \label{eq5}
    V(x,t)=H(t)\tan{\theta}=H(t)u_x(x,t).
\end{equation}

Looking back at equation (\ref{eq4}) and combining this with (\ref{eq5}) we obtain that

\begin{equation*} 
    V_x(x,t)=(H(t)u_x(x,t))_x.
\end{equation*}

Since $H(t)$ is independent of $x$, then this can be written as $V_x(x,t)=H(t)u_{xx}(x,t)$, which, by equation (\ref{eq4}), 
gives

\begin {equation} \label{eq7}
    H(t)u_{xx}(x, t)=\rho\Delta x u_{tt}(x,t).
\end{equation}

Since the motion of the sting is small, we can replace $H=T\cos{\theta}$ by $T$, as in \cite{BoyDiP}. Hence, (\ref{eq7}) takes the 
form of the one-dimensional wave equation:

\begin{equation} \label{wave}
    c^2u_{xx}=u_{tt}, 
\end{equation}

where $c^2=\frac{T}{\rho}$. We can easily verify that $c$ has the dimensions of 
velocity, as the tension $T$ is a force and $\rho$ is mass divided by length.

\subsection{Two-Dimensions} \label{twodim}
In the previous section we looked at how to model a musical instrument string, and we obtained the wave equation in 
one-dimension, (\ref{wave}). In this section, we will see what arguments we need to derive the second order wave equation.
In two-dimensions, the elastic string is now an elastic, flexible and homogeneous drumhead. Similarly to the string we looked at previously, 
the drumhead has vertical displacement only, i.e. there is no horizontal movement. Furthermore, the area of the membrane we act on is small
compared to the entire drumhead, and all angles of inclination are small, see \cite{Kr}.

\begin{figure}[h]
    \centering
    \includegraphics[scale=0.5]{images/grafic-5} 
    \caption{Drumhead membrane}
    \label{fig:2}
\end{figure}

Using \emph{Newton's Second Law}, as before, we obtain that the horizontal components of the tension $T$ are constant and independent of $x$.
Deflections of the membrane and angles are small, so the sides of the portions are small and approximately equal to $\Delta x$ and $\Delta y$, 
as shown in Figure \ref{fig:2}. The tension forces $T$ are acting on the sides of the portion, so it is approximately equal to $T\Delta x$ and $T\Delta y$, 
respectively. The forces are tangent to the membrane at all times as the membrane is entirely flexible.
\\

We will ignore horizontal movement entirely as, like in the one-dimensional case, the components are obtained by multiplying the cosines of 
the angles by the forces. The angles are very small, so we can assume that the cosine values are approximately 1 and the horizontal 
components of the forces at opposite ends of the membrane are approximately equal. This means that the motion is very small and we can ignore it.

The vertical movement is, however, more important than the horizontal one. From Figure \ref{fig:2} we can see that we have four different components of force.
Firstly, we have the components along the left and right sides given by $T\Delta y \sin\beta$ and $-T\Delta y\sin\alpha$, respectively. As the 
angles are very small, we can replace the sines by tangents, as in \cite{Kr}:

\begin{equation} \label{eq9}
\begin{split} 
    T\Delta y(\sin\beta - \sin\alpha) & \approx T\Delta y(\tan\beta-\tan\alpha)\\
    &= T\Delta y(u_x(x+\Delta x, y_1)-u_x(x, y_2)), 
\end{split}
\end{equation}

with $y_1, y_2 \in(y, y+\Delta y)$. Similarly, the vertical components in the other two sides are:

\begin {equation} \label{eq10}
    T\Delta x(u_y(x_1, y+\Delta y)-u_y(x_2,y)),
\end{equation}
where $x_1, x_2 \in(x, x+\Delta x)$.
\\

\emph{Newton's Second Law} with $\rho$ as the mass of undeflected membrane per unit area and $\Delta A=\Delta x\Delta y$ as the area of the portion
of the membrane when undeflected, gives the partial differential equation governing this motion:

\begin{equation} \label{eq11}
    \begin{split}
    T\Delta y\left[u_x(x+\Delta x, y_1)-u_x(x, y_2)]+T\Delta x[(u_y(x_1, y+\Delta y)-u_y(x_2,y)\right]\\
    =\rho\Delta x\Delta u_{tt}(x,y),
\end{split}
\end{equation}

evaluated at some suitable point $(\bar{x},\bar{y})$ corresponding to the section. Divide (\ref{eq11}) by $\rho\Delta x\Delta y$ to obtain:

\begin{equation} \label{eq12}
    \frac{T}{\rho}\left[\frac{u_x(x+\Delta x, y_1)-u_x(x,y_2)}{\Delta x}+\frac{u_y(x_1,y+ \Delta y)-u_y(x_2,y)}{\Delta y}\right]=u_{tt}.
\end{equation}

Letting $\Delta x$ and $\Delta y$ go to zero, both terms in the square bracket of (\ref{eq12}) are the limit definitions of derivatives with
respect to $x$ and $y$ respectively, and so we obtain the formula for the wave equation in two-dimensions:

\begin{equation} \label{wave2deq13}
    u_{tt}=c^2(u_{xx}+u_{yy}),
\end{equation}
where, as before, $c^2=\frac{T}{\rho}$. The wave equation in two-dimensions (\ref{wave2deq13}) can also be written as $u_{tt}=c^2\Delta u$, 
where $\Delta u=u_{xx}+u_{yy}$ is the \emph{Laplacian} of $u$ in two dimensions.

\subsection{Three Dimensions}
We can continue with similar arguments to derive the formula for the three-dimensional wave equation:

\begin {equation} \label{wave3deq14}
    u_{tt}=c^2(u_{xx}+u_{yy}+u_{zz}).
\end{equation}

This equation governs many physical processes and phenomena, such as the vibration of an elastic solid, 
sound waves in air, seismic waves propagating through the Earth, linearised supersonic airflow and electromagnetic waves such as light and radar, see
\cite{Str}

\section{Solution in One-Dimension}
In this section, we will prove the D'Alambert solution to the one-dimensional wave equation (\ref{wave}) 
in the absence of boundary conditions. The problem we wish to solve is as follows:

\begin{equation} \label{ivp1d}
    \begin{aligned}
    &u_{tt}-c^2u_{xx}=0, \quad \textrm{for} \quad x\in \mathbb{R},\quad t\in \mathbb{R}_{>0}\\
    &u(x,0)=g(x)\\
    &u_t(x,0)=h(x)
    \end{aligned}
\end{equation}

\subsection{General Form}

We first begin by showing that the solution has the form $u(x, t)=\phi(\xi)+\psi(\eta)$, where $\xi=x+ct$ and $\eta=x-ct$. This is quite straightforward
as the operator factorises nicely:

\begin{equation} \label{fact1d}
    u_{tt}-c^2u_{xx}=(\diffp[1]{}{t} - c \diffp[1]{}{x})(\diffp[1]{}{t} + c \diffp[1]{}{x})=0.
\end{equation}

If we denote the second bracket by $v$, we obtain that $v=u_t+cu_x$. Replacing this $v$ into (\ref{fact1d}), we obtain:

\begin{equation*} 
    (\diffp[1]{}{t} -c \diffp[1]{}{x})v=0, \quad \textrm{i.e.} \quad v_t-cv_x=0
\end{equation*}

We now have two first-order advection (or in some books, transport) equations:

\begin{align}
    v_t-cv_x=0 \label{veq}\\
    u_t+cu_x=v \label{ueq}
\end{align}

We will first focus on solving (\ref{veq}) using the method of characteristics, a mathematical technique that allows us to transform a partial differential
equation into a simple system of ordinary differential equation, see \cite{Ev}. The equation (\ref{veq}) is a homogeneous
advection equation. Consider $v(x(t),t)$ to be the value of $v$ at time $t$ and position $x(t)$. Then we can use the chain rule to differentiate
$v$ with respect to time:

\begin{equation} \label{diffv}
    \diff{v}{t}=\diffp[1]{v}{t}+\diffp[1]{v}{x}\diff{x}{t}.
\end{equation}

If we compare (\ref{veq}) and (\ref{diffv}) we see that $\diff{v}{t}=0$ if $\diff{x}{t}=-c$. This is our system of two simple ordinary differential 
equations. Solving them each at a time we see that the first one is equivalent to saying that $v$ is constant along $x(t)$ for any curve that 
solves the second ODE. We can integrate this and obtain that $x(t)=-ct+x_0$ where $x_0=x$ at time $t=0$ is just a constant. The equation $x(t)=-ct+x_0$
is the characteristic curve of this differential equation. If we consider the initial condition at $t=0$ in (\ref{ivp1d}) to be $v(x,0)=V(x)$, then the 
solution of (\ref{veq}) is $v(x,t)=V(x+ct)$. We can verify this is true by differentiating $v$ with respect to $t$ and $x$:

\begin{equation*}
    \begin{aligned}
    \diffp[1]{}{t}v(x(t), t)=cV'(x+ct)\\
    \diffp[1]{}{x}v(x(t), t)=V'(x+ct),
    \end{aligned}
\end{equation*}

which combined give us precisesly (\ref{veq}), and so $v(x,t)=V(x+ct)$ solves (\ref{veq}).
\\

Next, we want to solve (\ref{ueq}), where $v(x,t)=V(x+ct)$. This is now a nonhomogeneous advection equation. Let $w=\diffp[1]{u}{t}-c\diffp{u}{x}$, then 
$\diffp[1]{w}{t}+c\diffp{w}{x}=\diffp[2]{u}{t}-c^2\diffp[2]{u}{x}$. This is is precisesly the wave equation from (\ref{ivp1d}). Once again, we have two first
order partial differential equations:

\begin{align}
    \label{weq}
    w_t+cw_x=0\\
    \label{ueqw}
    u_t-cu_x=w.
\end{align}

We solve (\ref{weq}) in a similar manner to (\ref{veq}), by differentiating $w(x(t),t)$ with respect to time, using the chain rule:

\begin{equation} \label{diffw}
    \diff{w}{t}=\diffp[1]{w}{t}+\diffp[1]{w}{x}\diff{x}{t}. 
\end{equation}

Comparing with (\ref{weq}), we see that $\diff{w}{t}=0$ if $\diff{x}{t}=c$. As before, we get that the characteristic 
curve of this differential equation is $x(t)=ct+x_0$, where $x_0$ is $x$ at time $t=0$, and is just a constant. Hence, the solution of (\ref{weq})
is $w(x,t)=W(x-ct)$. We can verify that this solves the equation by differentiating.
\\

We are now close to the result we set off to prove. Consider $\frac{v+w}{2}$:

\begin{equation*}
    \begin{aligned}
    &\frac{v+w}{2}=\frac{1}{2}\left[\diffp[1]{u}{t}-c\diffp[1]{u}{x}+\diffp[1]{u}{t}+c\diffp[1]{u}{x}\right]=\diffp[1]{u}{t},\\
    &\textrm{i.e.} \quad \diffp[1]{}{t}u(x, t)=\frac{1}{2}\left[V(x+ct)+W(x-ct)\right].
    \end{aligned}
\end{equation*}

When integrating with respect to time, we obtain

\begin{equation} \label{int}
    u(x,t)=\frac{1}{2}\left[\int{V(x+ct)dt}+\int{W(x-ct)dt}\right].
\end{equation}

The first integral in (\ref{int}) is $\phi(x+ct)$ and the second one is $\psi(x-ct)$, such that $\phi'=-\frac{1}{2c}V$ and $\psi'=\frac{1}{2c}W$, respectively.
Therefore, the general solution to the one dimensional wave equation (\ref{ivp1d}) has the claimed form, 
\begin{equation} \label{gensol}
    u(x, t)=\phi(\xi)+\psi(\eta), \quad \textrm{where} \quad \xi=x+ct \quad \textrm{and} \quad \eta=x-ct.
\end{equation}

\subsection{Initial Value Problem}
We now have the general form (\ref{gensol}) of the solution to the one-dimensional wave equation from (\ref{ivp1d}). In this section, we will use it to derive the solution to
the full initial value problem from (\ref{ivp1d}), where the initial values are $u(x,0)=g(x)$ and $u_t(x,0)=h(x)$. First, set $t=0$ in (\ref{gensol}). This gives:

\begin{equation} \label{t=0}
    u(x,0)=\phi(x)+\psi(x)=g(x).
\end{equation}

Next, we differentiate (\ref{gensol}) using the chain rule, with respect to $t$ to obtain $u_t(x,t)=c\phi'(x+ct)-c\psi'(x-ct)$. Setting $t=0$ we get:

\begin{equation} \label{ut=0}
    u_t(x,0)=c\phi'(x)-c\psi'(x)=h(x).
\end{equation}

We simplify our calculations by changing the variable to something neutral, $s \in \mathbb{R}$. 
Differentiate (\ref{t=0}) to obtain:

\begin{equation} \label{eqq9}
    g'(s)=\phi'(s)+\psi'(s).
\end{equation}

Furthermore, divide (\ref{ut=0}) by $c$ to get:

\begin{equation} \label{eqq10}
    \frac{1}{c}h(s)=\phi'(s)-\psi'(s)
\end{equation}

We will continue by adding and subtracting (\ref{eqq9}) and (\ref{eqq10}):

\begin{align} \label{eqq11}
    \phi'(s)=\frac{1}{2}\Big(g'(s)+\frac{1}{c}h(s)\Big),\\
    \label{eqq12}
    \psi'(s)=\frac{1}{2}\Big(g'(s)-\frac{1}{c}h(s)\Big).
\end{align}

Integrating both (\ref{eqq11}) and (\ref{eqq12}) gives:

\begin{equation}
    \begin{aligned}
    \phi(s)=\frac{1}{2}g+\frac{1}{2c}\int^s_0h(n)dn+A\\
    \psi(s)=\frac{1}{2}g-\frac{1}{2c}\int^s_0h(n)dn+B,
    \end{aligned}
\end{equation}

where $A$ and $B$ are constants of integration.
\\

Equation (\ref{ut=0}) tells us that by adding (\ref{eqq11}) and (\ref{eqq12}) we have $A+B=0$ which means that $\phi$ and $\psi$ have the
same form, see \cite{Str}. Substituting $s=x+ct$ for $\phi(s)$ and $s=x-ct$ for $\psi(s)$ we obtain that

\begin{equation}
    u(x,t)=\frac{1}{2}g(x+ct)+\frac{1}{2c}\int^{x+ct}_0h(n)dn+\frac{1}{2}g(x-ct)-\frac{1}{2c}\int^{x-ct}_0h(n)dn.
\end{equation}

Rearranging gives the D'Alambert formula to the one-dimensional initial value wave equation, developed by the mathematician in 1746, see \cite{Str}:

\begin{equation} \label{DAla}
    u(x,t)=\frac{1}{2}\left[g(x+ct)+g(x-ct)\right]+\frac{1}{2c}\int^{x+ct}_{x-ct}h(n)dn.
\end{equation}

\subsection{An Application} \label{anapplication}
In this section, we want to present an application to the one-dimensional initial value wave equation. We introduce the boundary condition that
$u(0,t)=0$ to the problem in (\ref{ivp1d}) and obtain the boundary value problem:

\begin{equation} \label{bvp1d}
    \begin{aligned}
    &u_{tt}-c^2u_{xx}=0 \quad \textrm {for} \quad x \in \mathbb{R}, \quad t \in \mathbb{R}_{>0}\\
    &u(x,0)=g(x)\\
    &u_t(x,0)=h(x)\\
    &u(0,t)=0.
    \end{aligned}
\end{equation}

We know how to solve (\ref{ivp1d}), so we will transform (\ref{bvp1d}) into it by extending $u,g,h$ to all of $\mathbb{R}$ by odd reflection. See \cite{Ev} for more details on this method.
Hence, we define:

\begin{equation*}
    \begin{aligned}
        &\tilde{u}(x,t):=
        \begin{cases}
            u(x,t) \quad \textrm{when} \quad x \ge 0, t \ge 0\\
            -u(-x,t) \quad \textrm{when} \quad x \le 0, t \ge 0\\
        \end{cases}
        \\
        &\tilde{g}(x):=
        \begin{cases}
            g(x) \quad \textrm{when} \quad x \ge 0\\
            -g(-x) \quad \textrm{when} \quad x \le 0\\
        \end{cases}
        \\
        &\tilde{h}(x):=
        \begin{cases}
            h(x) \quad \textrm{when} \quad x \ge 0\\
            -h(-x) \quad \textrm{when} \quad x \le 0,\\
        \end{cases}
    \end{aligned}
\end{equation*}
\\

and so (\ref{bvp1d}) becomes 

\begin{align*}
    &\tilde{u}_{tt}=c^2\tilde{u}_{xx} \quad \textrm {for} \quad x \in \mathbb{R}, \quad t \in \mathbb{R}_{>0} \\
    &\tilde{u}(x,0)=\tilde{g}(x)\\
    &\tilde{u}_t(x,0)=\tilde{h}(x).\\
\end{align*}

This looks exactly like (\ref{ivp1d}), so we can use the D'Alambert formula (\ref{DAla}) to find a solution for $\tilde{u}$:

\begin{equation*}
    \tilde{u}(x,t)=\frac{1}{2}\left[\tilde{g}(x+ct)+\tilde{g}(x-ct)\right]+\frac{1}{2c}\int^{x+ct}_{x-ct}\tilde{h}(s)ds.
\end{equation*}

Recalling the definitions of $\tilde{u}, \tilde{g}$ and $\tilde{h}$ gives us the solution to (\ref{bvp1d}):

\begin{equation} \label{bvpsol}
    u(x,t)=
    \begin{cases}
        \frac{1}{2}\left[g(x+ct)+g(x-ct)\right]+\frac{1}{2c}\int^{x+ct}_{x-ct}h(s)ds \quad \textrm{if} \quad x \ge t \ge 0\\
        \frac{1}{2}\left[g(x+ct)-g(ct-x)\right]+\frac{1}{2c}\int^{x+ct}_{-x+ct}h(s)ds \quad \textrm{if} \quad 0 \le x \le t\\
    \end{cases}
\end{equation}

We will use the solution (\ref{bvpsol}) to the boundary value wave equation (\ref{bvp1d}) to solve the three-dimensional wave equation in 
the next section. Note that if $h \equiv 0$, then (\ref{bvpsol}) can be understood as a split in the initial displacement, one moving
to the right, and one to the left, see \cite{Ev}. Furthermore, this solution does not belong to $C^2$, the set of twice differentiable functions, 
unless $g''(0)=0$.

\section{Solution in Three-Dimensions}
In this section, we will prove the solution to the three-dimensional wave equation (\ref{wave3deq14}), as described in \cite[Chap. 2.4]{Ev}. To do this, we will use \emph{spherical
means}, which allow us to transform the partial differential equation into an integral over a sphere of radius $r$, centred at some point $\boldsymbol{x}$. 
Some advantages to using an integral form are that we obtain the solution on the whole domain and that we automatically take into account any
boundary conditions. For more information on spherical means, please see \cite{Sab}. 
\\
\\

The problem we wish to solve is

\begin{equation} \label{3deq}
\begin{aligned}
    &u_{tt}-c^2(u_{xx}+u_{yy}+u_{zz})=0 \quad \textrm {for} \quad x, y, z \in \mathbb{R}, \quad t \in \mathbb{R}_{>0}\\
    &u(x, y, z,0)=g(x,y,z)\\
    &u_t(x,y,z,0)=h(x,y,z).\\
\end{aligned}
\end{equation}

We intend, in this section, to derive the formula for $u$ in terms of $g$ and $h$. We plan to look at the average of $u$ as a
function of the time and radius over certain spheres, and apply the D'Alambert formula (\ref{bvpsol}) from Section \ref{anapplication}.
Note that throughout the next sections, we will let $\boldsymbol{x}=(x,y,z)$.

\subsection{Some Prerequisite Results} \label{prereq}
As we proceed with our proof of the three-dimensional wave equation, we require some other results, which we will not prove.
\\

Firstly, we introduce some notation to simplify our work in the upcoming sections. If we let $\boldsymbol{x}\in \mathbb{R}^n, t>0, r>0$, then we have:

\begin{equation} \label{averageU}
    U(\boldsymbol{x};r,t)=\dashint_{\partial B(\boldsymbol{x},r)}u(\boldsymbol{y},t)dS(\boldsymbol{y}),
\end{equation}

the average of $u(\cdot,t)$ over the sphere $\partial B(\boldsymbol{x},r)$, as in \cite{Ev}. Similarly, we have the averages of $g$ and $h$ over the same sphere:

\begin{equation} \label{averageGH}
    \begin{aligned}
        &G(\boldsymbol{x},r)=\dashint_{\partial B(\boldsymbol{x},r)}g(\boldsymbol{y})dS(\boldsymbol{y})\\
        &H(\boldsymbol{x},r)=\dashint_{\partial B(\boldsymbol{x},r)}h(\boldsymbol{y})dS(\boldsymbol{y}).
    \end{aligned}
\end{equation}

When we fix $\boldsymbol{x}$, we can regard $U$ as a function of just $r$ and $t$. Note that if $u$ satisfies equation
(\ref{3deq}), then we have $U \in C^m(\mathbb{R}_{\ge 0}\times[0,\infty))$, where $\mathbb{R}_{\ge 0}$ is the set of all the positive real numbers. The \emph{Euler-Poisson-Darbuox equation} is

\begin{equation} \label{EPDeq}
    \begin{aligned}
        &U_{tt}-c^2U_{rr}-\frac{n-1}{r}U_r=0 \quad \textrm {for} \quad r \in \mathbb{R}_{>0}, \quad t \in \mathbb{R}_{>0}, \quad n \in \mathbb{N}\\
        &U(r, 0)=G(r)\\
        &U_t(r,0)=H(r).
    \end{aligned}
\end{equation}

Another piece of mathematics we require for our derivation of the solution of the three-dimensional wave equation is the transformation formula for surface integrals. That is,
we want to show $\int_{\partial B(\boldsymbol{x},t)}f(\boldsymbol{y})dS=r^2\int_{\partial B(\boldsymbol{0},1)} f(\boldsymbol{x}+r\boldsymbol{z})dS(\boldsymbol{z})$. The change of variables formula in $\mathbb{R}^3$ from \cite{LooSter} is:

\begin{equation*}
    \int_{\partial B(\boldsymbol{x},r)}f(\boldsymbol{y})dS(y)=\int_U f(\boldsymbol{y}(s,t))\left\| \diffp[1]{}{s}\boldsymbol{y}(s, t) \times \diffp[1]{}{t}\boldsymbol{y}(s, t) \right \|ds dt
\end{equation*}

The function $f(\boldsymbol{y})$ can be written as $f(\boldsymbol{y})=f(\boldsymbol{x}+r(\frac{\boldsymbol{y}-\boldsymbol{x}}{r}))$ such that $\boldsymbol{y}(s,t)$ can be thought of as a parametrisation of any ball $\partial B(\boldsymbol{x},r)$ for $(s,t)\in U$. Then we can see
that $\frac{\boldsymbol{y}(s,t)-\boldsymbol{x}}{r}$ is a parametrisation of the sphere of radius 1, centred at the origin, $\partial B(\boldsymbol{0},1)$ for the same $(s,t)\in U$. Furthermore, using these parametrisations, we obtain

\begin{equation*}
    \left \| \diffp[1]{\boldsymbol{y}}{s} \times \diffp[1]{\boldsymbol{y}}{t} \right \| =r^2 \left \| \diffp[1]{}{s}\Big(\frac{\boldsymbol{y}-\boldsymbol{x}}{r}\Big) \times \diffp[1]{}{s}\Big( \frac{\boldsymbol{y}-\boldsymbol{x}}{r}\Big) \right \|.
\end{equation*}

Letting $\boldsymbol{z}(s,t)=\frac{\boldsymbol{y}(s,t)-\boldsymbol{x}}{r}$, we obtain our required result:

\begin{equation*}
    \begin{aligned}
    \int_U f(\boldsymbol{y}(s,t)) \left \| \diffp[1]{}{s}\boldsymbol{y}(s, t) \times \diffp{}{t}\boldsymbol{y}(s, t) \right \| dsdt&=r^2\int_U f(\boldsymbol{x}+r\boldsymbol{z}(s,t)) \left \| \diffp[1]{}{s}\boldsymbol{z}(s, t) \times \diffp[1]{}{t}\boldsymbol{z}(s, t) \right \| dsdt\\
    &=r^2\int_{\partial B(\boldsymbol{0},1)}f(\boldsymbol{x}+r\boldsymbol{z})dS(\boldsymbol{z}).
    \end{aligned}
\end{equation*}

\subsection{Kirchhoff's Formula}
We now want to derive the formula for the solution to the wave equation in three-dimensions (\ref{3deq}). We will assume that $u \in C^2(\mathbb{R}^3 \times
[0, \infty))$ solves (\ref{3deq}). Let us denote:

\begin{equation} \label{Udash}
    \tilde{U}:=rU
\end{equation}

\begin{equation} \label{GHdash}
    \begin{aligned}
        &\tilde{G}:=rG\\
        &\tilde{H}:=rH,    
    \end{aligned}
\end{equation}

where $U,G$ and $H$ are defined as in (\ref{averageU}) and (\ref{averageGH}), respectively. 
\\
\\
\\

Using this notation and the \emph{Euler-Poisson-Darboux equation} we can
obtain the boundary value problem solved in Section \ref{anapplication}:

\begin{equation*}
    \begin{aligned}
        \tilde{U}_{tt}&=rU_{tt}\\
        &=r\left[c^2U_{rr}+\frac{2}{r}U_r\right] \quad \textrm{by (\ref{EPDeq}) with} \quad n=3\\
        &=c^2rU_{rr}+2U_r\\
        &=(U+c^2rU_r)_r\\
        &=(c^2\tilde{U}_r)\\
        &=c^2\tilde{U}_{rr}.        
    \end{aligned}
\end{equation*}

This means that $\tilde{U}$ solves:

\begin{equation} \label{tilUwave}
    \begin{aligned}
        &\tilde{U}_{tt}-c^2\tilde{U}_{rr}=0 \quad \textrm {for} \quad r \in \mathbb{R}, \quad t \in \mathbb{R}_{>0}\\
        &\tilde{U}(r, 0)=\tilde{G}(r), \\
        &\tilde{U}_t(r, 0)=\tilde{H}(r),\\
        &\tilde{U}(0, t)=0.
    \end{aligned}
\end{equation}

Note that $\tilde{G}_{rr}(0)=0$ as $\tilde{U}(r,0)=\tilde{G}(r)$ in (\ref{tilUwave}). As this boundary value problem is the same as (\ref{bvp1d}), then we can apply 
the solution (\ref{bvpsol}) to (\ref{tilUwave}). Hence, for $0 \le r \le t$ we have 

\begin{equation} \label{newUtil}
    \tilde{U}(\boldsymbol{x};r,t)=\frac{1}{2c}\left[\tilde{G}(r+ct)-\tilde{G}(r-ct)\right]+\frac{1}{2c}\int^{r+ct}_{-r+ct}\tilde{H}(\boldsymbol{y})d\boldsymbol{y}.
\end{equation} 

Note that our definition of $U$ from (\ref{averageU}) implies that $u(\boldsymbol{x},t)=\lim_{r\rightarrow 0^+}U(\boldsymbol{x};r,t)$, so our definitions of $\tilde{G}, \tilde{H}$ in (\ref{GHdash}),
 and the new expression for $\tilde{U}$ from (\ref{newUtil}) lead to

\begin{equation} \label{almostK}
    \begin{aligned}
        u(\boldsymbol{x},t)&=\lim_{r\rightarrow 0^+}\frac{\tilde{U}(\boldsymbol{x};r,t)}{r}\\
        &=\lim_{r\rightarrow 0^+}\left[\frac{\tilde{G}(r+ct)-\tilde{G}(r-ct)}{2rc}+\frac{1}{2rc}\int^{r+ct}_{-r+ct}\tilde{H}(\boldsymbol{y})d\boldsymbol{y}\right]\\
        &=\tilde{G}'(t)+\tilde{H}(t)\\
        &=\diffp[1]{}{t}\left[t\dashint_{\partial B(\boldsymbol{x},t)}{gdS} \right]+t\dashint_{\partial B(\boldsymbol{x},t)}{hdS} \quad \textrm{by (\ref{averageGH})}
    \end{aligned}
\end{equation}
\\

Note that in section \ref{prereq} we saw that $\int_{\partial B(\boldsymbol{x},t)}f(\boldsymbol{y})dS=t^2\int_{\partial B(\boldsymbol{0},1)} f(\boldsymbol{x}+t\boldsymbol{z})dS(\boldsymbol{z})$. We will use this to complete our solution of the
three dimensional wave equation:

\begin{equation*}
    \begin{aligned}
    \diffp[1]{}{t}\dashint_{\partial B(\boldsymbol{x},t)} {g(\boldsymbol{y})dS} &= \dashint_{\partial B(0,1)} {Dg(\boldsymbol{x}+tz)zdS(z)}\\
    &=\dashint_{\partial B(\boldsymbol{x},t)}{Dg(\boldsymbol{y})(\frac{\boldsymbol{y}-\boldsymbol{x}}{t})dS(\boldsymbol{y})},
    \end{aligned}
\end{equation*}

where $D$ represents the first-order derivative with respect to time. Substituting this into (\ref{almostK}), we obtain the solution to the three-dimensional wave equation, which is known as \emph{Kirchhoff's formula}, but is 
actually, due to Poisson, see \cite{Str}:

\begin{equation} \label{KircE}
    u(\boldsymbol{x},t)=\dashint_{\partial B(\boldsymbol{x},t)} th(\boldsymbol{y})+Dg(\boldsymbol{y})\Big(\frac{\boldsymbol{y}-\boldsymbol{x}}{t}\Big)+g(\boldsymbol{y})dS(\boldsymbol{y}).
\end{equation}

\subsection{Another Formula}
\emph{Strauss}, in \cite{Str} gives another form of Kirchhoff's formula for the solution to the wave equation in three-dimensions. In this section, we aim to prove that 
this formula is the same as the one we have seen in (\ref{KircE}). The formulation mentioned in \cite{Str} is as follows, where $\boldsymbol{x}=(x,y,z)$, as before:

\begin{equation} \label{KircSTR}
    u_S(\boldsymbol{x}, t)=\frac{1}{4 \pi c^2 t}\iint_{\partial B(\boldsymbol{x}, t)} h(\boldsymbol{y}dS(\boldsymbol{y}))+\diffp[1]{}{t}\Big( \frac{1}{4 \pi c^2 t}\iint_{\partial B(\boldsymbol{x}, t)} g(\boldsymbol{y})dS(\boldsymbol{y})\Big).
\end{equation}

For a more comfortable identification of the two formulae by different authors, we will rewrite (\ref{KircE}): 

\begin{equation} \label{KircEV}
    u_E(\boldsymbol{x},t)=\dashint_{\partial B(\boldsymbol{x},t)} th(\boldsymbol{y})+Dg(\boldsymbol{y})\Big(\frac{\boldsymbol{y}-\boldsymbol{x}}{t}\Big)+g(\boldsymbol{y})dS(\boldsymbol{y}).
\end{equation}

In order to start identifying common parts between the two formulae, recall that $\dashint_{\partial B(\boldsymbol{x}, t)}$ was the average over the sphere $\partial B(\boldsymbol{x}, t)$.
This is the same as $\frac{1}{4 \pi c^2 t^2}\iint_{\partial B(\boldsymbol{x}, t)}$, since the surface area of the sphere is $4\pi r^2$, where $r$ is the radius of the sphere, 
in our case, $t$. Using this, we can rewrite (\ref{KircEV}) as:

\begin{equation*}
    \begin{aligned}
    u_E(\boldsymbol{x},t)&=\iint_{\partial B(\boldsymbol{x},t)} th(\boldsymbol{y})+Dg(\boldsymbol{y})\Big(\frac{\boldsymbol{y}-\boldsymbol{x}}{t}\Big)+g(\boldsymbol{y})dS(\boldsymbol{y})\\ 
    &=\iint_{\partial B(\boldsymbol{x},t)} th(\boldsymbol{y})dS(\boldsymbol{y})+\iint_{\partial B(\boldsymbol{x},t)} Dg(\boldsymbol{y})\Big(\frac{\boldsymbol{y}-\boldsymbol{x}}{t}\Big)+g(\boldsymbol{y})dS(\boldsymbol{y}).
    \end{aligned}
\end{equation*}

Our aim is to show that $u_S$ and $u_E$ are the same. We start by letting $g=0$ in both (\ref{KircEV}) and (\ref{KircSTR}):

\begin{equation} \label{g=0}
    \begin{aligned}
        u_S(\boldsymbol{x}, t)&=\frac{1}{4\pi c^2 t}\iint_{\partial B(\boldsymbol{x}, t)}h(\boldsymbol{y})dS(\boldsymbol{y})\\
        &=\frac{1}{4\pi c^2 t^2}\iint_{\partial B(\boldsymbol{x}, t)}th(\boldsymbol{y})dS(\boldsymbol{y})\\
        &=u_E(\boldsymbol{x}, t).
    \end{aligned}
\end{equation}

This computation shows that the first parts of the two formulae are the same. Next, we show that the second parts are the same by letting $h=0$:

\begin{equation} \label{starr}
    \begin{aligned}
        u_S(\boldsymbol{x},t)&=\diffp[1]{}{t}\frac{1}{4\pi c^2 t}\iint_{\partial B(\boldsymbol{x}, t)}g(\boldsymbol{y})dS(\boldsymbol{y})\\
        &=\diffp[1]{}{t}t\Big(\frac{1}{4\pi c^2 t}\iint_{\partial B(\boldsymbol{x}, t)}g(\boldsymbol{y})dS(\boldsymbol{y})\Big)\\
        &=\diffp[1]{}{t}t\Big(\dashint_{\partial B(\boldsymbol{x}, t)}g(\boldsymbol{y})dS(\boldsymbol{y})\Big).
    \end{aligned}
\end{equation}

Using the transformation of surface integrals described in section \ref{prereq}, we obtain that $\dashint_{\partial B(\boldsymbol{x}, t)}g(\boldsymbol{y})dS(\boldsymbol{y})=\dashint_{\partial B(\boldsymbol{0}, 1)}g(\boldsymbol{x}+t\boldsymbol{z})dS(\boldsymbol{z})$. 
This means that (\label{starr}) becomes:
 
\begin{equation} \label{h=0} 
    \begin{aligned}
        u_S(\boldsymbol{x}, t)&=\diffp[1]{}{t}t\Big(\dashint_{\partial B(\boldsymbol{0}, 1)}g(\boldsymbol{x}+t\boldsymbol{z})dS(\boldsymbol{z})\Big)\\
        &=\dashint_{\partial B(\boldsymbol{0}, 1)}g(\boldsymbol{x}+t\boldsymbol{z})dS(\boldsymbol{z})+t\diffp[1]{}{t}\dashint_{\partial B(\boldsymbol{0}, 1)}g(\boldsymbol{x}+t\boldsymbol{z})dS(\boldsymbol{z})\\
        &=\dashint_{\partial B(\boldsymbol{0}, 1)}g(\boldsymbol{x}+t\boldsymbol{z})dS(\boldsymbol{z})+t\dashint_{\partial B(\boldsymbol{0}, 1)}Dg(\boldsymbol{x}+t\boldsymbol{z}) \cdot \boldsymbol{z}dS(\boldsymbol{z})\\
        &=\dashint_{\partial B(\boldsymbol{x}, t)}g(\boldsymbol{y})dS(\boldsymbol{y})+t\dashint_{\partial B(\boldsymbol{x}, t)}Dg(\boldsymbol{y})\Big(\frac{\boldsymbol{y}-\boldsymbol{x}}{t}dS(\boldsymbol{y})\Big)\\
        &=\dashint_{\partial B(\boldsymbol{x}, t)}g(\boldsymbol{y})dS(\boldsymbol{y})+t\dashint_{\partial B(\boldsymbol{x}, t)}Dg(\boldsymbol{y})(\boldsymbol{y}-\boldsymbol{x})dS(\boldsymbol{y})\\
        &=u_E(\boldsymbol{x},t).
    \end{aligned}
\end{equation}

Since in both (\ref{g=0}) and (\ref{h=0}) we showed that $u_S=u_E$, then we can conclude that the two solutions to the three-dimensional wave equation are the same.

\section{Solution in Two-Dimensions}
In this section, we will present the solution to the two-dimensional wave equation derived in \ref{twodim}. Before we prove the solution, we will discuss \emph{Huygens' 
Principle} and explain why we solved the three-dimensional equation first.

\subsection{Huygens' Principle}
\emph{Huygens' Principle} describes how we can physically hear sharp sounds and see sharp images. \emph{Folland}, in \cite{Fol}, gives a very simple example to illustrate 
this. If we are in a dark room at a position $x_0$ that doesn't change, and we set off a flash light at the origin of the three-dimensional system we 
created from the room, at time $t=0$, then we will be able to see the light only for as long as it takes it to travel from its position to where we are, the room
becoming dark again after. Another simple example to illustrate \emph{Huygens' Principle} is given by \emph{Strauss} in \cite{Str}, where he looks at how the sound
of a musical instrument is heard by the human ear. Sound is carried through air at precisely the fixed speed $c$ without any disturbance by assuming
there are no walls or other inhomogeneities in the air, so that at any time $t$, the listener hears exactly what notes the musical instrument plays at the time $t-\frac{d}{c}$,
where $d$ is the distance between the listener and the instrument, rather than hearing a mixture of notes from earlier echoing over the new notes. These three-dimensional 
examples show phenomena that are known as \emph{Huygens' Principle}, which asserts that an initial state is observed at a different place as an effect that is 
very rigorously delimited, see \cite{Hil} .
\\

This physical phenomenon makes sense in three-dimensions, and it is encountered in everyday life. However, we might be interested in what happens if a two-dimensional system would follow
the same rules. This does not make sense, and an example to show this comes from dropping a pebble into an undisturbed pond. The waves that are created by the pebble satisfy 
approximately well the two-dimensional wave equation with a certain speed $c$, where $x$ and $y$ are horizontal coordinates of the 2D plane. If we assume there exists
a waterbug without any movement, at a distance $d$ from the point of impact between the pebble and the water, it will feel a wave at time $t=\frac{d}{c}$, but that wave will
not stop upon reaching the waterbug but continue to send out waves. Those secondary waves become so small that they are no longer felt after a period of time, but they 
do not actually ever stop, as described in \cite{Str}. This means that \emph{Huygens' Principle} no longer holds. This is because the time does not limit the initial state is not limited 
in time, i.e. once a signal reaches a point in the two-dimensional space, it continues there indefinitely. More details about the physical interpretations can be found in \cite{Hil}.
\\

\emph{Huygens's Principle} allowed us to straightforwardly derive the solution of the three-dimensional equation using spherical means. However, this same principle
means that the same method cannot be employed to solve the two-dimensional equation, but it offers an alternative based on the three-dimensional problem. The trick is
to solve the case in three-dimensions and then \emph{descend} into two dimensions by letting the $z$-coordinate in $u_{tt}=c^2(u_{xx}+u_{yy}+u_{zz})$ equal $0$, as in \cite{Ev}. We shall see how this works in the 
next section. 

\subsection{Poisson's Formula}

In this section, we will prove the solution to the two-dimensional wave equation. We cannot transform the \emph{Euler-Poisson-Darboux equation} (\ref{EPDeq}) using (\ref{Udash}) and (\ref{GHdash}) 
into the one-dimensional equation when $n=2$. This is why we do not have a specific method to solving the two-dimensional problem other than considering it as the previously solved three-dimensional problem
with the third spatial coordinate equal to $0$. Let $u \in C^2(\mathbb{R}^2 \times [0, \infty))$ solve the wave equation in two-dimensions:

\begin{equation} \label{2d}
    \begin{aligned}
        &u_{tt}-c^2(u_{xx}+u_{yy})=0 \quad \textrm{with} \quad x,y \in \mathbb{R}, \quad t\in \mathbb{R}_{>0}\\
        &u(x, y, 0)=g(x,y),\\
        &u_t(x,y,0)=h(x,y).
    \end{aligned}
\end{equation}

Then, we can let $\bar{u}(x, y, z, t):=u(x, y, t)$ such that the problem we wish to solve becomes:

\begin{equation} \label{2das3d}
    \begin{aligned}
        &\bar{u}_{tt}-c^2(\bar{u}_{xx}+\bar{u}_{yy}+\bar{u}_{zz})=0 \quad \textrm{with} \quad x,y,z \in \mathbb{R}, \quad t\in \mathbb{R}_{>0}\\
        &\bar{u}(x, y, z, 0)=\bar{g}(x,y,z), \quad \textrm{where} \quad \bar{g}(x,y,z)=g(x,y)\\
        &\bar{u}_t(x,y,z,0)=\bar{h}(x,y,z), \quad \textrm{where} \quad \bar{h}(x,y,z)=h(x,y).
    \end{aligned}
\end{equation}

Let $\boldsymbol{x}=(x,y)\in \mathbb{R}^2$ and $\boldsymbol{\bar{x}}=(x,y, 0)\in \mathbb{R}^3$. Then, combining (\ref{2das3d}) with the (\ref{almostK}) form of Kirchhoff's formula, we obtain:

\begin{equation} \label{star}
    \begin{aligned}
        u(\boldsymbol{x},t)&=\bar{u}(\boldsymbol{\bar{x}}, t)\\
        &=\diffp[1]{}{t}\Big(t\dashint_{\partial \bar{B}(\boldsymbol{\bar{x}}, t)} \bar{g} d\bar{S}\Big)+t \dashint_{\partial \bar{B}(\boldsymbol{\bar{x}}, t)} \bar{h}d\bar{S},
    \end{aligned}
\end{equation}

where $\partial\bar{B}(\boldsymbol{\bar{x}},t)$ is the ball centered at $\boldsymbol{\bar{x}}$, of radius $t>0$, in $\mathbb{R}^3$, and that $d\bar{S}$ is the two-dimensional measure on 
$\partial\bar{B}(\boldsymbol{\bar{x}},t)$. Note that $\dashint_{\partial \bar{B}(\boldsymbol{\bar{x}}, t)} \bar{g} d\bar{S}=\frac{1}{4\pi t^2}\int_{\partial \bar{B}(\boldsymbol{\bar{x}}, t)}\bar{g} d\bar{S}$, as they
both represent the average over the sphere. Let $\gamma (\boldsymbol{y})=(t^2-|\boldsymbol{y}-\boldsymbol{x}|^2)^{1/2}$, for $\boldsymbol{y} \in B(\boldsymbol{x}, t)$. Using 
this we obtain:

\begin{equation*}
    \begin{aligned}
        \dashint_{\partial \bar{B}(\boldsymbol{\bar{x}}, t)} \bar{g} d\bar{S}&=\frac{1}{4\pi t^2}\int_{\partial \bar{B}(\boldsymbol{\bar{x}}, t)}\bar{g} d\bar{S}\\
        &=\frac{2}{4 \pi t^2} \int_{B(\boldsymbol{x}, t)} g(\boldsymbol{y})(1+|D\gamma (\boldsymbol{y})|^2)^{1/2}d\boldsymbol{y}.
    \end{aligned}
\end{equation*}

We have $2$ as the numerator of the fraction as the sphere $\partial\bar{B}(\boldsymbol{\bar{x}}, t)$ consists of two hemispheres. Furthermore, $D\gamma (\boldsymbol{y})$ is the first derivative of $\gamma$ with respect to $\boldsymbol{y}$, and it is
$D\gamma (\boldsymbol{y})=-\frac{|\boldsymbol{y}-\boldsymbol{x}|}{(t^2-|\boldsymbol{y}-\boldsymbol{x}|^2)^{1/2}}$. 
\\
\\

Then,

\begin{equation*}
    \begin{aligned}
        (1+|D\gamma (\boldsymbol{y})|^2)^{1/2}&=\Big(1-\Big(\frac{|\boldsymbol{y}-\boldsymbol{x}|}{(t^2-|\boldsymbol{y}-\boldsymbol{x}|^2)^{1/2}}\Big)^2\Big)^{1/2}\\
        &=\Big(1+\frac{|\boldsymbol{x}-\boldsymbol{y}|^2}{t^2-|\boldsymbol{y}-\boldsymbol{x}|^2}\Big)^{1/2}\\
        &=\Big(\frac{t^2}{t^2-|\boldsymbol{y}-\boldsymbol{x}|^2}\Big)^{1/2}\\
        &=t(t^2-|\boldsymbol{y}-\boldsymbol{x}|^2)^{-1/2}.
    \end{aligned}
\end{equation*}

So, our integral becomes:

\begin{equation*}
    \begin{aligned}
        \dashint_{\partial \bar{B}(\boldsymbol{\bar{x}}, t)} \bar{g} d\bar{S}&=\frac{1}{2 \pi t^2}\int_{B(\boldsymbol{x},t)}\frac{tg(\boldsymbol{y})}{(t^2-|\boldsymbol{y}-\boldsymbol{x}|^2)^{1/2}}d\boldsymbol{y}\\
        &=\frac{1}{2 \pi t}\int_{B(\boldsymbol{x},t)}\frac{g(\boldsymbol{y})}{(t^2-|\boldsymbol{y}-\boldsymbol{x}|^2)^{1/2}}d\boldsymbol{y}\\
        &=\frac{t}{2}\dashint_{B(\boldsymbol{x},t)}\frac{g(\boldsymbol{y})}{(t^2-|\boldsymbol{y}-\boldsymbol{x}|^2)^{1/2}}d\boldsymbol{y}.
    \end{aligned}
\end{equation*}

Hence, (\ref{star}) is: 

\begin{equation} \label{triangle}
    \begin{split}
        u(\boldsymbol{x},t)=\frac{1}{2}\diffp[1]{}{t}\Big(t^2\dashint_{B(\boldsymbol{x},t)}\frac{g(\boldsymbol{y})}{(t^2-|\boldsymbol{y}-\boldsymbol{x}|^2)^{1/2}}d\boldsymbol{y}\\
        +\frac{t^2}{2}\dashint_{B(\boldsymbol{x},t)}\frac{h(\boldsymbol{y})}{(t^2-|\boldsymbol{y}-\boldsymbol{x}|^2)^{1/2}}d\boldsymbol{y}.
    \end{split}
\end{equation}

Using a similar transformation to the one described in section \ref{prereq}, we get that $t^2\dashint_{B(\boldsymbol{x},t)}\frac{g(\boldsymbol{y})}{(t^2-|\boldsymbol{y}-\boldsymbol{x}|^2)^{1/2})}d\boldsymbol{y}$ is equal to  $t\dashint_{B(0,1)}\frac{g(\boldsymbol{x}+t\boldsymbol{z})}{(1-|\boldsymbol{z}|^2)^{1/2}}d\boldsymbol{y}$. 
Therefore, the first part of (\ref{triangle}) can be written as:

\begin{equation*}
    \begin{aligned}
        \diffp[1]{}{t}\Big(t^2\dashint\frac{g(\boldsymbol{y})}{(t^2-|\boldsymbol{y}-\boldsymbol{x}|^2)^{1/2})}d\boldsymbol{y}\Big)&=\dashint_{B(0,1)}\frac{g(\boldsymbol{x}+t\boldsymbol{z})}{(1-|\boldsymbol{z}|^2)^{1/2}}d\boldsymbol{z}+t\dashint_{B(0,1)}\frac{Dg(\boldsymbol{x}+t\boldsymbol{z})\cdot \boldsymbol{z}}{(1-|\boldsymbol{z}|^2)^{1/2}}d\boldsymbol{z}\\
        &=t\dashint_{B(\boldsymbol{x},t)}\frac{g(\boldsymbol{y})}{t^2-|\boldsymbol{y}-\boldsymbol{x}|^2)^{1/2}}d\boldsymbol{y}+t\dashint_{B(\boldsymbol{x},t)} \frac{Dg(\boldsymbol{y})\cdot (\boldsymbol{y}-\boldsymbol{x})}{t^2-|\boldsymbol{y}-\boldsymbol{x}|^2)^{1/2}}d\boldsymbol{y}.
    \end{aligned}
\end{equation*}

We can now rewrite $u(\boldsymbol{x},t)$ from (\ref{triangle}) in the form of the solution of the two-dimensional wave equation, also known as Poisson's formula, see \cite{Ev}.
 
\begin{equation*}
    u(\boldsymbol{x},t)=\frac{1}{2}\dashint_{B(\boldsymbol{x},t)} \frac{tg(\boldsymbol{y})+t^2h(\boldsymbol{y})+tDg(\boldsymbol{y})\cdot (\boldsymbol{y}-\boldsymbol{x})}{t^2-|\boldsymbol{y}-\boldsymbol{x}|^2)^{1/2}}d\boldsymbol{y}.
\end{equation*}

\section{Solutions in Higher Dimensions}

\subsection{Odd Dimensions}
\subsection{Even Dimensions}
\subsection{Uniqueness of Solutions}

\section{Conclusion}

\newpage 

\bibliographystyle{unsrt}
\bibliography{bibliography}

\end{document}
